\begin{description}
  \item[Bose-Gas bei $T=0$] 
    %
    \begin{align*}
      \Braket{\hat{n}_{s,\vec{k}}} = \begin{cases}
        0 & \quad\forall\, \vec{k} \neq 0 \\
        \ge 0 & \quad\forall\, \vec{k} = 0 , \quad \sum_{s}^{} \Braket{\hat{n}_{s,\vec{k}=0}} = N
      \end{cases} 
    \end{align*}
    %
    Das bedeutet der Grundzustand ist makroskopisch besetzt. Man nennt das
    Bose-Einstein-Kondensat. 
    %
    \begin{align*}
      \frac{N_0}{N} > 0 && N \to \infty 
    \end{align*}
    %
    Man kann daraus sehr leicht die Fluktuation der Teilchenzahlen berechnen.
    Wenn man das mit der Großkanonischen Beschreibung vergleicht, dann muss
    man um diesen Effekt zu bekommen das folgende machen
\end{description}
%
\begin{align*}
  \frac{1}{e^{\beta(\epsilon(\vec{k}) - \mu)} - 1} = \begin{cases}
    0 & \quad\forall\, \epsilon(\mathcal{k}) > \mu \\
    \infty & \text{ für } \epsilon(\vec{k} = 0) = \mu
  \end{cases} 
\end{align*}
%
Wir erreichen dass wenn das chemische Potential den festen Wert des Grundzustands
hat. Wir wollen die Funktion oben invertieren, also sollte die Funktion
$\mu(N)$ umkehrbar sein. Das ist im allgemeinen jedoch nicht möglich. Man braucht
eine Funktion die varriiert in abhängigkeit der Parameter. Physkialisch gesehen, 
was passiert da? Sobald es eine endliche Temeperatur gibt dann ist das Kondensat
gelöst. Dann kann man das großkanonische ensemble verwenden. in 3 dimensionen
ist das im allgeminen nicht der Fall. Wir können diesen Fall mit unserer berechnung
nicht berücksichtigen. Wir müssen also die gleichung korrigieren. Unser 
startpunkt war dabei die Summe oben. Wir hatten geschrieben
%
\begin{align*}
  N = \sum_{s, \vec{k}}^{} \Braket{\hat{n}_{s,\vec{k}}} = \sum_{n}^{} \Braket{\hat{n}_{s,\vec{k}=0}}
  + \sum_{s, \vec{k} \neq 0 }^{} \Braket{\hat{n}_{s,\vec{k}}}.
\end{align*}
%
Wenn man die Summe durch ein Integral ersetzt dann spielt ein Punkt keine rolle
und man kann wieder das Ergebnis von oben verwenden. Sie sehen, dass nun die
Zustandsgleichung ein bisschen anders ist als Vorher
%
\begin{align*}
  N = N_0 + \frac{v(2S+1}{\lambda_T^3} g_{\frac{3}{2}} (Z)
\end{align*}
%
Wir müssen nun diese Gleichung näher untersuchen. Für
%
\begin{align*}
  N_0 = 
  \begin{cases}
    0 &  T > T_c(\rho) \\
    N - \frac{V(2 S + 1}{\lambda_T^3} g_{\frac{3}{2}}(1) &  T < T_c(\rho)\\
  \end{cases} 
\end{align*}
%
%
\begin{align*}
  \frac{N_0 }{N} = 1 - \frac{\rho_c(T) }{\rho} = 1- \left( \frac{T}{T_c(\rho)} \right)^{\frac{3}{2}}
\end{align*}
%
Dies ist der Anteil des Kondensats.
%
\begin{align*}
  1  - \frac{N_0}{N} 
\end{align*}
%
Ist der nicht kondensierte Anteil. Wir können nun kurz analysieren was das
eigentlich bedeutet. Man hat ein Kondensat oberhalb der kritischen dichte.
Es verschwindet jedoch genau wenn die Dichte diesen Wert erreicht. Man 
kann das graphisch illustrieren. \\
Ausrduck Name vorname matrikelnummer\\
elektronisch name nachnahme \\
sprechzeiten 
\begin{table}[h!]
  \centering
  \begin{tabular}{c c c c c c}
     & & & nächste woche \\
    Mi & 11-12 & Mi & & vorlesungsfrei nach absprache \\
    & 14-16:30 & & 14-18 \\
    Do & 10-12 & Do & & \\
         &      &  & 14-16 \\
  \end{tabular}
\end{table}

% TODO: skizze
\subsection*{Freies ideales Fermi-Gas}
\emph{Schwabl Kapitel 4.2, 4.3}
%
\begin{align*}
  \epsilon(\vec{k}) = \frac{\hbar^2}{2 m} \vec{k}^2 && s = -S, -S + 1, \ldots , S
  && \text{ Entartung: } 2 S + 1 && S = \frac{1}{2} , \frac{3}{2}, \ldots
\end{align*}
%
Damit ist
%
\begin{align*}
  Z(\epsilon \ge 0) = \frac{2 S + 1) ((2 \pi m)^{\frac{d}{2}}}{k^d}
  \frac{1}{\Gamma\left( \frac{d}{2} \right)} \epsilon^{\frac{d}{2} - 1} \\
  \text{ Fermi-Dirac-Verteilung } \Braket{ \hat{n}_{\vec{k},s}} = 
  \frac{1}{e^{\beta(\epsilon(\vec{k}) - \mu)} + 1}
\end{align*}
%
Wir berechnen zuerst großkanonisch die Teilchenzahl
%
\begin{align*}
  N(T, \mu, V) = \sum_{\vec{k},s}^{} \Braket{\hat{n}_{\vec{k},s}} = 
  V \int_{0}^{ + \infty} \d{\epsilon} \frac{Z(\epsilon) 1}{e^{\beta(\epsilon-\mu)} + 1}
\end{align*}
%
Wir müssen dieses Integral analysieren indem wir für $Z$ diese besondere Form
verwenden.
%
\begin{align*}
  N(T, \mu, V)= \frac{V(2 S + 1}{\lambda_T^d} f_{\frac{d}{2}}(Z)
\end{align*}
%
mit der Fugazität $Z = e^{\beta \mu}$ und $\lambda_T = \sqrt{2 \pi m k_B T}
\frac{1}{h}$. 
%
\begin{align*}
  f_\alpha(Z) = \int_{0 }^{ + \infty} \d{x} x^{\alpha-1} \frac{1}{\frac{1}{z} e^x + 1} 
  \frac{1}{\Gamma(\alpha)} = 
  \sum_{m=1}^{\infty} (-1)^{n+1} \frac{z^n}{n^\alpha}
\end{align*}
%
%
\begin{align*}
  U(T, \mu, V) & = \sum_{\vec{k},s}^{} \epsilon(\vec{k}) \Braket{\hat{n}_{\vec{k},s}} \\
               & = V \int_{0}^{ + \infty} \d{\epsilon} Z(\epsilon) 
  \frac{\epsilon}{e^{\beta(\epsilon-\mu)} + 1} \\
  &  = V ( 2 S + 1) 
  \frac{1}{\lambda_T^d} f_{\frac{d}{2} + 1} (z) 
  \frac{d}{2} k_B T
\end{align*}
%
%
\begin{align*}
  \Omega(T, \mu, V) & = - k_B T \ln{Z_{GK}} = 
  - k_B T \sum_{\vec{k},s}^{} \ln{\left( 1 + e^{-\beta(\epsilon(\vec{k}) - \mu)} \right)} \\
  & = - k_B T V \int_{0}^{+ \infty} \d{\epsilon} Z(\epsilon) \ln{\left[ 1 + e^{-\beta(\epsilon- \mu)} \right]} \\
  & = - k_B T V (2  S + 1)
  (2 \pi m )^{\frac{d}{2}} \frac{1}{h^d} \underbrace{\frac{1}{\Gamma\left( \frac{d}{2} \right)}
  \int_{0}^{+ \infty} \d{\epsilon} \epsilon^{\frac{d}{2} - 1}\ln{
\left[ 1 + e^{- \beta (\epsilon - \mu)} \right]}}_{ = (k_B T) ^{\frac{d}{2}}
\frac{1}{\Gamma\left( \frac{d}{2} \right)}  \int_{0}^{\infty} x^{\frac{d}{2} - 1}
\ln{\left[ 1 + Z e^{-x} \right]}}
\end{align*}
%
%
\begin{align*}
  \frac{1}{\Gamma(\frac{d}{2})} \int_{0}^{ \infty} x^{\frac{d}{2} - 1}
  \ln{\left[  1 + Z e^{-x} \right]} = 
  \frac{1}{\Gamma\left( \frac{d}{2} \right)} \frac{Z}{d} x^{\frac{d}{2}}
  \ln{\left[ 1 + Z e^{-x} \right]} |_0^\infty
  - \frac{1}{\Gamma\left( \frac{d}{2} \right)} \int_{0}^{\infty}
  \d{x} \frac{2}{d} x^{\frac{d}{2}} 
  \underbrace{\frac{-z e^{-x}}{1 + z e^{-x}}}_{=  - \frac{1}{e^{x} \frac{1}{z} + 1}}
 \end{align*}
%
%
\begin{align*}
  \Omega(T, \mu, V) = k_B T V \left( 2 S + 1 \right) 
  \frac{1}{\lambda_T^d} \frac{2}{d} f_{\frac{d}{2} + 1}(z) = 
  - \frac{2}{d} U
\end{align*}
%
Wir können nun den Druck berechnen, da wir das großkanonische Potential ausgerechnent
haben
%
\begin{align*}
  P & = - \left( \pd{\Omega}{V} \right)_{T, \mu} = 
  - \frac{\Omega}{V} = \frac{2}{d} \frac{U}{V} 
  \implies U = \frac{PV d}{2}
\end{align*}
%
Im Grenzfall ist
%
\begin{align*}
  Z \ll 1 && \iff && \beta \mu \to -\infty && \iff && - \mu \gg k_B T
\end{align*}
%
%
\begin{align*}
  f_\alpha(z) = z - z^{-\alpha} z^2 + \mathcal{O}(z^3)
\end{align*}
%
Daraus folgt
%
\begin{align*}
  P V = \frac{2}{d} U \simeq V (2 S + 1) 
  \frac{1}{\lambda_T^d} k_B T \left[ z - z^{-\left( \frac{d}{2} + 1 \right)
  z^2 + \mathcal{O}(z^3)} \right]
\end{align*}
%
%
\begin{align*}
  N \simeq V (2 S + 1) 
  \frac{1}{\lambda_T^d} \left[ z - z^{-\frac{d}{2}} z^2 z^2 + \mathcal{O}(z^3) \right] (*)
\end{align*}
%
Das bedeutet
%
\begin{align*}
  \frac{P V }{N k_B T} & = 
  \frac{z - z ^{\left( - \frac{d}{2} + 1 \right) z^2 + \mathcal{O}(z^3)}}{
    (z - z ^{\left( - \frac{d}{2} \right) z^2 + \mathcal{O}(z^3)}) } \\
    & = 1 - \left( z^{-\left( \frac{d}{2} + 1\right)} - z^{-\frac{d}{2}} \right)Z
    + \mathcal{O} \\
    & = 1 + z^{-\left( \frac{d}{2} + 1 \right)} z + \mathcal{O}(z^2) \quad (+)
\end{align*}
%
%
\begin{align*}
  Z = \frac{N}{V} \frac{1 }{2 S + 1} \lambda_T^d 
\end{align*}
%
Aus $(*)$ folgt
%
\begin{align*}
  Z = \frac{N}{V} 
  \frac{1}{2 S + 1} \lambda_T^d
\end{align*}
%
und dann mit $(+)$
%
\begin{align*}
  PV = N k_B T \left[ 1 + z^{-\left( \frac{d}{2} + 1 \right)} 
  \frac{\rho }{2 S + 1} \lambda_T ^d \right]
\end{align*}
%
Dies ist die erste Quantenkorrektur zur Zustandsgleichung des idealen Gases für
Fermionen. Wenn man betrachtet, wann dies Gleichung erfüllt ist:
Wir haben angenommen dass $z \ll 1$ also
%
\begin{align*}
  \rho \lambda_T^d \ll 1 && T \to \infty && S \to \infty
\end{align*}
%
Bevor wir den Grenzfall $ z \gg 1 $ betrachten untersuchen wir den Fall
$T = 0$. 
%
\begin{align*}
  \Braket{\hat{n}_{s ,\vec{k}}} = \frac{1}{e^{\beta(\epsilon(\vec{k}) - \mu)} + 1}
  \underset{=}{T\to0} \begin{cases}
    0 & \text{ für } \epsilon(\vec{k}) > \mu \\
    1 & \text{ für } \epsilon(\vec{k}) < \mu \\
  \end{cases} 
\end{align*}
% TODO: Skizze
%
%
\begin{align*}
  \text{ Fermi-Energie } \epsilon_F = \mu \text{ für } T = 0\\
  \text{ Fermi-Temperatur } \epsilon_F = k_B T_F
\end{align*}
%
Das schöne an diesem Grenzfall ist, dass die Rechnung sehr einfach wird.
%
\begin{align*}
  N & = \sum_{\vec{k},s}^{} \Braket{\hat{n}_{\vec{k},s}} = 
  V \int_{0}^{+ \infty} \d{\epsilon} Z(\epsilon) \Omega(\epsilon_F - \epsilon) \\
  & = V (2 S + 1)
  (2 \pi m)^{\frac{d}{2}}\frac{1}{h^d} 
  \frac{1}{\Gamma\left( \frac{d}{2} \right)} \underbrace{\int_{0}^{\epsilon_F}
  \epsilon^{\frac{d}{2} - 1} \d{\epsilon}}_{ = \frac{2}{d} \epsilon_F^{\frac{d}{2}}}
\end{align*}
%
%
\begin{align*}
  \epsilon_F = \left( \rho
  \Gamma\left( \frac{d}{2} + 1 \right)/ (2S + 1) \right)^{\frac{2}{d}}
  \frac{h^2 }{2 \pi m}
\end{align*}
%
Beispiel 
%
\begin{align*}
  d = 3 && \epsilon_F 
  \left( \frac{6 \pi^2}{2 S + 1} \rho \right)^{\frac{2}{3}} 
  \frac{\hbar^2}{2 m}
\end{align*}
%
%
\begin{align*}
  E_0 = \sum_{\vec{k},s}^{} \epsilon(\vec{k}) \Braket{\hat{n}_{\vec{k},s}}
= U(T = 0)  = N \epsilon_F
\frac{\frac{d}{2}}{\frac{d}{2} + 1}
\end{align*}
%
%
\begin{align*}
  d = 3 && E_0 = \frac{3}{5} N \epsilon_F
\end{align*}
%
%
\begin{align*}
  \text{ Grenzfall } z \gg 1 && \iff && \beta \mu \gg 1
  && \iff && \mu > k_B T
\end{align*}
%
%
\begin{align*}
  d = 3 && \text{ Sommerfeld-Entwicklung } \equiv \text{ asymptotische Entwicklung 
der } f_\alpha(z)
\end{align*}
%
%
\begin{align*}
  f_{\frac{3}{2}} (z) = \frac{4}{3} \frac{1}{\sqrt{\pi}}
  (\ln{ z}) ^{\frac{3}{2}} \left[ 1 + \frac{\pi^2}{8}
  \frac{1}{(\ln{z})^2} + \ldots \right]
\end{align*}
%
Das funktioniert auch für die zweite Funktionen die wir brauchen, 
für die Energie haben wir
%
\begin{align*}
  f_{\frac{5}{2}} (z) = \frac{8}{15} \frac{1}{\sqrt{\pi}} (\ln{z})^{\frac{5}{2}}
  + \pi^{\frac{3}{2}}\frac{1}{3} (\ln{z})^{\frac{1}{2}} + \ldots 
\end{align*}
%
Sie können diese Formel verwenden um die Teilchenzahl als funktion von 
$z$ zu entwickeln. Damit ergibt sich
%
\begin{align*}
  \mu & = \epsilon_F \left[  1- \frac{\pi^2}{12} \left( \frac{T}{T_F} \right)^2
  + \mathcal{O}\left( \left( \frac{T}{T_F}^4 \right) \right)\right] \\
  U & = \frac{3}{5} N \epsilon_F \left[ 1 + \frac{5}{12} \pi^2 \left( \frac{T }{T_F} \right)^{2} 
+ \mathcal{O}\left( \left( \frac{T}{T_F} \right)^4 \right)\right]
\end{align*}
%
Wenn $z \gg 1$ dann ist $T \ll T_F$ und wir haben ein entartetes Fermi-Gas.
Man kann jetzt die Wärmekapazität berechnen:
%
\begin{align*}
  C_{V, N} = \dd{U}{T} = \frac{N  k_B \pi^2}{2} \frac{T }{T_F} \xrightarrow{T\to 0} 0
\end{align*}
%
% TODO: skizze

Beispiele sind 
% \begin{itemize}
%   % \item Elektronen im Metall $ \epsilon_F = \SI{1-10}{\electronvolt}$ $T_F = \SI{e4-e5}{\Kelvin}$
%   % \item Weiße Zwerge $\epsilon_F = \SI{1-10}{\electronvolt}$ $T_F = \SI{3e9}{\Kelvin}$
%   % \item \ce{He3}$ \epsilon_F = \SI{4e-4}{\electronvolt}$ $T_F = \SI{5}{\kelvin}$ theoretisch, experimentell \SI{1}{\Kelvin}.
% \end{itemize}







