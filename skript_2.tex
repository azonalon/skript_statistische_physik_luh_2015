\subsection*{Anwendungen und Rechenmethoden}

\begin{itemize}
    \item Reale Gase und Fl\"ussigkeiten
    \item Ideale Quantengase
    \item Magnetismus
\end{itemize}
\begin{table}[h!]
    \centering
    \begin{tabular}{p{5cm} | p{5cm}}
        Ideales Gas & Reales Gas \\
        \hline
        1 Art von Teilchen & Gemisch (L\"osung) \\
        Punktteilchen & Endliche Ausdehnung, Innere Struktur, Spaltung und Bindung \\
        Keine Wechselwirkung & Wechselwirkung \\
    \end{tabular}
\end{table}
\subsection*{Ideales Molek\"ulgas}
Ideal ist hier im sinne von ``Keine Wechselwirkung'' gemeint. Wir betrachten 
also ein verd\"unntes Gas.
%
\begin{align*}
    \mathcal{H} = \bigotimes_{i=1}^{N} \mathcal{H}_i^{(1)} &&
    H = \sum_{i=1}^{N} H_i^{(1)} && H_i^{(1)} = 
    I_1 \otimes I_2 \otimes \ldots \otimes H_i \otimes I_{n-1} \otimes \ldots 
\end{align*}
%
%
\begin{align*}
    [H_i^{(1)} , H_i^{(1)}]  = 0
\end{align*}
%

%
\begin{align*}
    Z_N & = \trace_{\mathcal{H}^{(N)}} e ^{-\beta H(N)} = 
    \trace_{\mathcal{H}^{(N)}} \prod_{i=1}^{N} e ^{-\beta H_i^{(1)}} \\
    & = \trace_{\mathcal{H}_1^{(1)}}\trace_{\mathcal{H}_2^{(1)}} \ldots \trace_{\mathcal{H}_N^{(1)}} 
    \prod_{i=1}^{N} e^{-\beta H^{1}_i} = \prod_{i=1}^{N} \trace_{\mathcal{H}_i^{(1)}} = 
    \left( \trace_{\mathcal{H}_i^{(1)}} e ^{-\beta H^{(1)}} \right)^{N}
\end{align*}
%
%
\begin{align*}
    F = -k_B T \ln{Z_N} = - k T N \ln{Z_1} && \implies &&  
    \frac{F}{N} = - k_B \ln{Z_1}
\end{align*}
%
\subsection*{Freiheitsgrade eines Molek\"uls}
\begin{itemize}
    \item Bewegung des Schwerpunkts
    \item Rotation
    \item Schwingung (Vibrationen)
    \item Elektronische Anregung
\end{itemize}
Wir nehmen an, dass die Freiheitsgrade s\"amtlich unabh\"angig sind.
%
\begin{align*}
    H^{(1)} & = H_{\text{SP}} + H_{\text{rot}} + H_{\text{vib}} +
    H_{\text{Elek}} \\
    \mathcal{H}^{(1)} & = \mathcal{H}_{\text{sp}} \otimes \mathcal{H}_{\text{rot}} \otimes
    \mathcal{H}_{\text{vib}} \otimes \mathcal{H}_{\text{El}} \\
    Z_1 & = \trace e^{-\beta H^{(1)}} = 
    \left( \trace_{\mathcal{H}_{\text{sp}}} e^{- \beta H_{\text{sp}}} \right)
    \left( \trace_{\mathcal{H}_{\text{rot}}} e^{- \beta H_{\text{rot}}} \right)
    \left( \trace_{\mathcal{H}_{\text{vib}}} e^{- \beta H_{\text{vib}}} \right)
    \left( \trace_{\mathcal{H}_{\text{el}}} e^{- \beta H_{\text{el}}} \right)
\end{align*}
%
Damit ist
%
\begin{align*}
    f = - k_B T \ln{ Z_{\text{sp}}} - k_B T \ln{Z_{\text{rot}}}
- k_B T \ln{Z_{\text{vib}}}
- k_B T \ln{Z_{\text{el}}}
= f_{\text{SP}} + f_{\text{rot}} + f_{\text{vib} } + f_{\text{el}}
\end{align*}
%
Beispiel
% imap <c-s> \text{}<left><left><left><left><left>
%
\begin{align*}
    \frac{C_{V, N}}{N} & = \frac{1}{N} 
    \frac{\delta Q }{ \d{T}} = - \frac{T }{N} \left( \pd{^2 F}{T^2} \right)_{V,N}
    = - T \left( \pd{^2f}{T^2} \right)_{V, N} \\
    & = C_{\text{sp}} + C_{\text{rot}} + C_{\text{vib}} + C_{\text{El}}
\end{align*}
%
%
\begin{align*}
    P = - \left( \pd{F}{V} \right)_{T, N} \simeq - N \left( \pd{f_{\text{sp}}}{V} \right)_{T, N}
\end{align*}
%
% TODO: Diagramm
%
\begin{align*}
    T_0 & = T_F \text{ (Fermi Temperatur) } \\
        & = T_C \text{ (BEC kritische Temperatur)}\\
\end{align*}
%
%
\begin{align*}
    T_0 < \SI{5}{\kelvin} \text{ f\"ur } \ce{He^3}, \ce{He^4}
\end{align*}
%
\subsection*{Rotationsanteil}
F\"ur ein zweiatomiges Moek\"ul aus 2 verschiedenen Atomen zb
\ce{HCl} aber nicht \ce{O2}, \ce{N2}
%
\begin{align*}
    H_\text{rot} = \frac{\vec{L}^2}{2 I}
\end{align*}
%
Mit dem Tr\"agheitsmoment $I = m_\text{reduziert} R^2$
%
\begin{align*}
    \epsilon_{lm} = \frac{\hbar^2}{2 I} l(l+1), \quad\forall\, l \in \N,
    m = -l, -l+1 ,\dotsc, +l
\end{align*}
%
%
\begin{align*}
    \Theta_\text{rot} = \frac{\hbar^2}{I k_B}
\end{align*}
%
%
F\"ur $T \ll \Theta_\text{rot}$:
\begin{align*}
    C_\text{rot} \approx k_B 3 \left( \frac{\Theta_\text{rot}}{T} \right)^2
    e^{- \frac{\Theta_\text{rot}}{T}} \xrightarrow{ T \to 0} 0 
\end{align*}
%
F\"ur hohe Temperaturen $T \gg \Theta_\text{Rot}$ gilt dann
%
\begin{align*}
    C_\text{rot} \approx k_B \left[ 1 + \frac{1}{180} \left( \frac{\Theta_\text{rot}}{T} \right)^2 + \ldots \right] \xrightarrow{T \to \infty} k_B
\end{align*}
%
% TODO: Diagram
Beispiel
%
\begin{align*}
    \ce{H Cl} && \Theta_\text{rot} & = \SI{30}{\kelvin} \\
    \ce{H2} && &= \SI{170}{\kelvin} \\
    \ce{O2} && &= \SI{4}{\kelvin} \\
\end{align*}
%
\subsection*{Vibrationsanteil}
Beispiel: Harmonische Schwingungen eines zweiatomigen Molek\"uls.
Aus der Eigenfrequenz $\omega$ kann man eine Temperaturskala definieren.
%
\begin{align*}
    \Theta_\text{vib} = \frac{\hbar W }{k_B}
\end{align*}
%
Beispiele
%
\begin{align*}
    \ce{H Cl} && \Theta_\text{vib} & = \SI{4e3}{\kelvin} \\
    \ce{H2} && &= \SI{6e3}{\kelvin} \\
    \ce{O2} && &= \SI{2e3}{\kelvin} \\
\end{align*}

%
\begin{align*}
    C_\text{vib} = k_B \left( \frac{\Theta_\text{vib}}{T} \right)^2
    \frac{1}{\left[ 2 \sinh\left( \frac{\Theta_\text{vib}}{2} \right) \right]}
    = \begin{cases}
        \left( \frac{\Theta_\text{vib}}{T} \right)^2 e^{\frac{\Theta_\text{vib}}{T}} \to 0 & T \ll \Theta_\text{vib} \\
        1 - \frac{1}{12}\left( \frac{\Theta_\text{vib}}{T} \right)^2 + \ldots \to 1 & T \gg \Theta_\text{vib}
    \end{cases} 
\end{align*}
%
% TODO: Diagram

\subsection*{Elektronischer Anteil}
Wasserstoffatom
%
\begin{align*}
    \mathcal{E}_n = - \frac{\SI{13.6}{\electronvolt}}{n^2}
\end{align*}
%
Die minimale Anregungsenergie ist
%
\begin{align*}
    \mathcal{E}_2 - \mathcal{E}_1 = \SI{10}{\electronvolt} \approx \SI{1e5}{\kelvin}
\end{align*}
%
% TODO: BIG Diagram
\section*{Reales Klassisches Gas Teil I}
\emph{Schwabl Kapitel 5.3}
\begin{table}[h!]
  \centering
  \begin{tabular}{c c c c c c}
    Klassisch & Verd\"unnt und hohe Temperaturen \\
    Real & Wechselwirkende Teilchen (Atome oder Molek\"ule)
  \end{tabular}
\end{table}
%
\begin{align*}
  Z_K(N) & = \frac{1}{h^{3N}} 
  \frac{1}{N!} \int_{V^N}^{} \d{^{3N}} r \int_{\R^{3N}}^{} e ^{-\beta H (\vec{r}, \vec{p})} \\
  H(\vec{r},\vec{p}) & = \sum_{i=1}^{N} 
  \frac{\vec{P}_1^2}{2m} + \frac{1}{2} \sum_{i,j=1}^{N} w (\abs{\vec{r}_1 - \vec{r}_2})
\end{align*}
%
Dabei ist $w(r)$ ein Zentralpotential.
% TODO: Skizze
%
\begin{align*}
  Z_K(N) = \frac{1}{N! h^{3N}} ( \int_{\R^{3N}}^{} d^{3N}p e^{-\beta \sum_{i}^{} 
  \frac{\vec{P}_i^2}{2m}} 
)\end{align*}
%
Der Konfigurationsanteil ist:
%
\begin{align*}
  Q(T,V, N) = \int_{V^N}^{} \d{^{3N}r} e^{-\beta \sum_{i=1}^{}w (\abs{\vec{r}_i,\vec{p}_i})}
\end{align*}
%
Die thermische Wellenl\"ange ist:
%
\begin{align*}
  \lambda_T = \frac{h}{\sqrt{2 \pi m k_B T}}
\end{align*}
%
F\"ur $w=0$ haben wir ein ideales Gas.
%
\begin{align*}
  Q = V^N
\end{align*}
%
Wenn $w \neq 0$ ist es ein reales Gas, wir k\"onnen dann $Q$ im allgemeinen
nicht berechnen. Es gibt jedoch N\"aherungsverfahren
\begin{itemize}
  \item Virialentwicklung
  \item Molekularfelfdn\"aherung
  \item Computersimulation
  \item Molek\"uldynamik
  \item Monte Carlo
\end{itemize}

Potentiale $w(r)$ f\"ur neutrale Teilchen.
Lennard-Jones-Potential
%
\begin{align*}
  w(r) = 4 \epsilon \left[ \left( \frac{\sigma}{r} \right)^{12} - \left( \frac{\sigma}{r} \right)^6 \right], \quad\forall\, \epsilon, \sigma > 0
\end{align*}
%
% TODO: skizze
Teilchen mit einem harten Kern, also harte Kugeln:
%
\begin{align*}
  w(r) = \begin{cases}
    \infty & r < d \\
    - w_r & d < r \\
    0 r > l \\
  \end{cases} 
\end{align*}
%
\subsection*{Virialentwicklung}
Die Gro\ss{}kanonische Zustandssumme
%
\begin{align*}
  Z_\text{GK} = \sum_{N=0}^{\infty} Z_K(N) e^{\beta \mu N}
\end{align*}
%
%
\begin{align*}
  Z_K(0) = 1, && Z_K(1) = \frac{V}{\lambda_T^3}, && Z_K(2) = ? 
\end{align*}
%
Klassisches Gas: Fugazit\"at $e^{\beta\mu} = y \ll 1$, das bedeutet
$-\mu \gg k_B T$. Mithilfe dieser Annahme k\"onnen wir einfach eine Taylor-Entwicklung
machen
%
\begin{align*}
  Z_\text{GK} = Z_K(0) + Z_K(1) y + Z_K(2)y^2 + \mathcal{O}(y^3) \\
\end{align*}
%
Das bedeutet 
%
\begin{align*}
  \Omega(T, V, \mu) = - k_B T \ln{ Z_\text{GK}} = -k_B T \left[ 
  \frac{V}{\lambda_T^3} y + \left( Z_K(2) - \frac{V}{2 \lambda^3_T} \right) y ^2 
+ \mathcal{O}(y^3)\right]
\end{align*}
%
So erh\"alt man
%
\begin{align*}
  PV = N k_B T \left[ 1+ B \rho + \mathcal{O}(\rho^2) \right] && \rho= \frac{N}{V} \ll 1
\end{align*}
%
$B$ ist der 2. Virialkoeffizient. 
%
\begin{align*}
  B = \frac{V}{2} - \frac{\lambda_T^6 }{V} Z_K(2)
\end{align*}
%
%
\begin{align*}
  Z_K(2) = \frac{1}{\lambda_T^6} 
  \frac{1}{2} \int_{V}^{} \d{^3 r_1} \int_{V}^{} \d{^3r_2} e^{-\beta w(\abs{\vec{r}_1 - \vec{r}_2})}
\end{align*}
%
Daraus folgt dann
%
\begin{align*}
  B = - \frac{1}{2V} \int_{V}^{} \d{^3 r_1} \int_{V}^{} \d{^3 r_2} 
  \left[ e^{-\beta w ( \abs{\vec{r}_1 - \vec{r}_2})} - 1 \right] && \left( \frac{V}{N} \right)^{\frac{1}{3}} \gg \sigma, l
\end{align*}
% TODO: Skizze
%
\begin{align*}
  W \simeq 0 \iff (e^{-\beta w} \approx 1) \quad\forall\, \vec{r}_2 \notin V
  \text{ und } \vec{r}_1 \in V
\end{align*}
%
Wir k\"onnen also das Integral auf den ganzen Raum erweitern und
nicht nur \"uber das Volumen.
%
\begin{align*}
  \vec{r} = \vec{r}_2 - \vec{r}_1 \implies B = - \frac{1}{2V} \int_{V}^{} \d{^3 r_1}
  \int_{\R^3}^{} \d{^3 r} \left[ e^{-\beta w(r)} - 1 \right]
\end{align*}
%
Wir haben einen Kugelsymmetrischen Integranden also verwenden wir Kugelkoordinaten.
%
\begin{align*}
  B = - 2 \pi \int_{0}^{\infty} \d{r} r^2 \left[  e^{-\beta w(r)} - 1 \right]
\end{align*}
%

%
\begin{align*}
  B & = 2 \pi \int_{0}^{d} \d{r} (r^2 e ^{+ \beta w_s} - 1 ) + 0 \\
    & = \frac{2 \pi}{3} d^3 (l^3  - d^3) \left(\e^{\beta w_s} - 1 \right)
  \simeq - \frac{a}{k_B T} + b
\end{align*}
%
mit %
\begin{align*}
  a & = \frac{2\pi}{3} (l^3 - d^3) w_s > 0 \text{ (Anziehende Wechselwirkung)}\\ 
  b & = \frac{4 \cdot 4 \pi}{3} \left( \frac{d}{2} \right)^3 > 0
\end{align*}
%
Man sieht, dass $b$ eigentlich nur 4-mal das Volumen einer Kugel ist.
%
\begin{align*}
  P V = N k_B T \left[ 1 + \left(  - \frac{a}{k_B T} + b \right) 
  \frac{N}{V}\right] + \mathcal{O}\left( \left( \frac{N}{V} \right)^2 \right) && (*)
\end{align*}
%
\subsection*{Van-der-Waals Zustandsgleichung}
%
\begin{align*}
  \left[ P + a\left( \frac{N}{V} \right)^2 \right] ( V - N b) = N k_B T
\end{align*}
%
F\"ur
%
\begin{align*}
  \frac{N}{V} \ll 
  \frac{1}{b} \iff
  \frac{V}{N} \gg b \implies
  (*)
\end{align*}
\subsection*{Harte Kugeln}
%
\begin{align*}
  w(r) = \begin{cases}
    \infty & r < \sigma \\
    0 & r > \sigma \\
  \end{cases} 
\end{align*}
%
%
\begin{align*}
  Q_{HK} = \int_{V'}^{} \d{^{3N}r} = \underset{N \gg 1}{ = } (V - N b)^N
\end{align*}
%
ist das Volumen von $V'$
%
%
\begin{align*}
  V' = \left\{ \{\vec{r}_1\} \in V^N \subset \R^{3N} 
  \text{ mit } \abs{\vec{r}_1 - \vec{r}_{j}} > \sigma \quad\forall\,
 i\neq j = 1, \ldots , N\right\}
\end{align*}
%
$b$ ist gr\"o\ss{}er oder gleich dem Volumen des harten Kerns
$\frac{\pi \sigma^3}{6}$.
Dies ist kein mathematischer Beweis, aber
\begin{itemize}
  \item Exakt in $D=1$
  \item Numerischer Beweis f\"ur $D = 2, 3$
\end{itemize}
Sie m\"ussen das als ein asymptotishes Verhalten
verstehen wenn $N $ sehr gr\"o\ss{} ist. Wenn sie das mit der virialentwicklung
vergleichen dann sehen sie, dass
%
\begin{align*}
  b = 4 \cdot \left( \frac{\pi \sigma^3 }{b} \right)
\end{align*}
%
Das ist Physik, mann kann nicht immer alles mathematisch streng beweisen.  Wir
haben jetzt einen spezialfall diskutiert und werden uns nun allgemeineren
probemen widmen. Wir nehmen an, dass die WEchselwirkung die folgende Struktur hat
%
\begin{align*}
  w(\vec{r}) = \begin{cases}
    w_K(r) > 0 \text{ f\"ur } r  < \sigma \\
    w_K(r) \le 0 \text{ f\"ur } r  > \sigma \\
  \end{cases} 
\end{align*}
%
Wir nehmen an, dass
%
\begin{align*}
  w_K(r) \gg k_B T \gg \abs{ w_F (r)}
\end{align*}
%
\begin{align*}
  w(r) = w_K(r) + w_F(r)
\end{align*}
%
mit 
%
\begin{align*}
  w_K(r > \sigma) &= 0 \\
  w_F(r \le \sigma)& = 0 \\
\end{align*}
%

%
% TODO: Skizze
%
\begin{align*}
  Q = Q_H \frac{\int_{V^N}^{} \d{^{3N}r} e^{-\beta \sum_{i<j}^{} w_K \left( \vec{r}_2  - \vec{r}_1  \right) }   e ^{-\beta \sum_{ i<j }^{} w_F (\vec{r}_2 - \vec{r}_1)}   }{\int_{V^N}^{} \d{^{3N}r} e ^{-\beta \sum_{i=1}^{w_K\left( \vec{r}_1 -\vec{r}_2 \right)}}}
  = Q_\text{HK} \Braket{ e ^{- \beta \sum_{i < j}^{} w_F (\vec{r}_1 - \vec{r}_j)}}_{HK}
\end{align*}
%
$\Braket{ \ldots }_{HK}$ ist der Erwartungswert im Gas harter Kueln
\begin{description}
  \item[Definition] Dichte
    %
    \begin{align*}
      \rho(\vec{r}) = \sum_{i=1}^{N} \delta(\vec{r} - \vec{r}_i) &&
      \int_{\Delta V}^{} \rho (\vec{r}) \d{^3 r} = \sum_{\vec{r}_1 \in
      \Delta V}^{} 1
    \end{align*}
    %
    Dies entspricht der Teilchenzahl in $\Delta V$.
\end{description}

%
\begin{align*}
  \sum_{i < j = 1}^{N} w_F (|\vec{r}_i - \vec{r}_j|) = 
  \frac{1}{2} \int_{V}^{} \d{^3} R_1 \int_{V}^{} \d{^3} R_2 
  w_F (\abs{\vec{R}_1 - \vec{R}_2}) \rho(\vec{R}_1) \rho(\vec{R}_2)
\end{align*}
%
%
\begin{align*}
  Q & = Q_{HK} \Braket{ \exp{
    \left[ - \frac{\beta}{2} \int_{V}^{} \d{^3R_1} \int_{V}^{} \d{^3 R_2}
w_F \left( \abs{\vec{R}_1- \vec{R}_2}  \right) \rho(\vec{R}_1)\rho(\vec{R}_2)\right]}}_{HK} \\
& = Q_{HK}  \exp{
    \left[ - \frac{\beta}{2} \int_{V}^{} \d{^3R_1} \int_{V}^{} \d{^3 R_2}
w_F \left( \abs{\vec{R}_1- \vec{R}_2}  \right) \rho_{MF}(\vec{R}_1)\rho_{MF}(\vec{R}_2)\right]}_{HK}
\end{align*}
%
mit
%
\begin{align*}
  \rho_{MF}(\vec{R}) = \Braket{ \rho(\vec{R})}_{HK}
\end{align*}
%
% TODO: skizze


\subsection*{Reales Klassisches Gas}
Molekularfeldn\"aherung, Van-Der-Waals-Zustandsgleichung
%
\begin{align*}
  Z = \frac{1}{N!} \lambda_T^{-3N} Q, && Q & = Q_{HK} 
  \Braket{\exp{\left[ -\beta \sum_{i < j}^{} w_F(\abs{\vec{r}_i - \vec{r}_j}) \right]}}_{HK} \\
  &&& = Q_{HK} \Braket{\exp{\left[ - \beta \int_{V}^{} \d{^{3}R_1} \int_{V}^{} \d{^3 R_2}
  W(\abs{\vec{R}_1 - \vec{R}_2}) \rho(\vec{R}_1) \rho(\vec{R}_2) \right]}}_{HK} \\
  &&& = Q_{HK} \exp{\left[ -\frac{\beta}{2} \int_{V}^{ } \d{^3 R_2} w_F \left(\abs{\vec{R}_1 - \vec{R}_2}\right) \rho_{MF}(\vec{R}_1) \rho_{MF}(\vec{R}_2) \right]}
\end{align*}
%
wobei %
\begin{align*}
  \rho_{MF} (\vec{R}) = \Braket{ \rho (\vec{R})}_{HK}
\end{align*}
%
Die Molekularfeldn\"aherung ist die n\"aherung des mittleren Feldes, 
\emph{mean-field approximation}.

Die Potentialenergie eines Teilchens $i$ lautet
%
\begin{align*}
  V_1 (\vec{r}) = \sum_{j \neq i, j = 1}^{}w_F (\abs{\vec{r}_i - \vec{r}_j})
  \simeq \int_{V}^{} \d{^3 R}w_F (\abs{\vec{r}_1 - \vec{R}}) \rho(\vec{R})
\end{align*}
%
%
\begin{align*}
  V_{MF}(\vec{r}) = \Braket{V_i(\vec{r})}_{HK} = 
  \int_{V}^{} \d{^3 R } w_F (\abs{\vec{r} - \vec{R}}) \rho_{MF} (\vec{R})
\end{align*}
% TODO: skizze
%
\begin{description}
  \item[1. Ansatz] Symmetrieerhaltung: Es gibt keinen Grund f\"ur das Gas
    eine bestimmte Richtung zu bevorzugen, es gibt also auch keinen Grund
    anzunehmen, dass die dichte nicht konstant ist. Wir nehmen sie also als
    Konstant an.
    %
    \begin{align*}
      \rho_{MF}(\vec{R}) = \frac{N}{V}
    \end{align*}
    %
    Daraus folgt dann
    %
    \begin{align*}
      Q = Q_{HF} \exp \left[ - \frac{\beta}{2} \underbrace{\int_{V}^{} \d{^3 R_1} \int_{V}^{} \d{^3 R_2}
      w_F (\abs{\vec{R}_1 - \vec{R}_2})}_{= - 2 a V}\left( \frac{N}{V} \right)^2\right]
    \end{align*}
    %
    Wir k\"onnen das wie bei der Virialentwicklung behandeln: Wir werden das
    deswegenen nicht nochmal explizit berechnen. Die gr\"o\ss{}e $a \ge 0 $ ist
    extensiv.
    %
    \begin{align*}
      Z = \frac{1}{N!} \lambda_T ^{-3N} \left[ V - Nb \right]^N
      \e^{+ \frac{\beta N^2}{V} a}
    \end{align*}
    %
    Damit ist dann
    %
    \begin{align*}
      F = - k_B T \ln{Z} = - k_B T N \left[ 1 + \ln{\left( 
      \frac{V - Nb}{N \lambda_T ^{3}}\right)} \right] - \frac{a N^2 }{V}
    \end{align*}
    %
    (Dies ist der Startpunkt der Computeraufgabe 2)
    Wir werden hier die Physik und die idee hinter der Berechnung vermitteln, 
    das eigentliche Berechnen ist Teil der Computeraufgabe 2.
    %
    \begin{align*}
      \mu(T, v) &= - k_B T \left[ \ln{\frac{v-b}{\lambda_T^3}} + 
      \frac{v}{b - v}\right] - \frac{2a}{v} \\
      P(T, v) & = \frac{k_B T }{v - b} - \frac{a}{v}
    \end{align*}
    %
    Wir erkennen, dass diese Gleichung nur von dem spezifischen Volumen und ..
    abh\"angt.
    %
    \begin{align*}
      v = \frac{V}{N} = \rho^{-1}
    \end{align*}
    %
    Die isotherme kompressibilit\"at ist
    %
    \begin{align*}
      \kappa_T = - \frac{1}{V} \left( \pd{V}{P} \right)_{N,T}
    \end{align*}
    %
    daraus folgt
    %
    \begin{align*}
      \left( P+ a \left(\frac{N}{V}\right)^2\right) (V - N b) = k_B T N 
    \end{align*}
    %
    % TODO: Isothermen im $P-v$ Diagramm
    %
    \begin{align*}
      \kappa_T \ge 0 \implies T \ge T_c (\rho) = 2 a \rho (1- b\rho)^2 \\
      & \implies \rho_c = \frac{1}{3b}, T_c = T(\rho_c) = \frac{8}{27} 
      \frac{a}{k_B b} \implies P_c = \frac{a}{27 b^2} 
    \end{align*}
    %
  \item[2. Ansatz] Symmetriebrechung
    Wir haben ein heterogenes System, in welchem 2 Phasen koexistieren.
    Gas und Fl\"ussigkeit. Das bedeutet ein Teil des Systems ist in 
    Gasform, ein anderer Teil jedoch in Fl\"ussiger Form
    %
    \begin{align*}
      \text{Gas: } T, N_G, V_G \quad \implies \quad F (T, N_G, V_G) ,
      v_G = \frac{V_G}{N_G}, \mu(T, v_G), P(T, v_G)
    \end{align*}
    %
    F\"ur die Fl\"ussigkeit ist es \"ahnlich:
    %
    \begin{align*}
      T, N_F, V_F \quad \implies \quad F(T, N_F, V_F), v_F = \frac{V_F}{N_F}, 
      \mu(T, v_F) , P(T, v_F)
    \end{align*}
    %
    Wir nehmen an, dass die WEchselwirkung zwischen Gas und Fl\"ussigkeit
    vernachl\"assigbar ist:
    %
    \begin{align*}
      F(T, V_G, V_F, N_G, N_F) &= F(T, V_G, N_G) + F(T, N_F, V_F) \\
      N & = N_G + NF (*)\\
      V &= V_G + V_F (*)
    \end{align*}
    %
    % TODO: Skizze
    Im thermodynamischen Gleichgewicht ist
    \begin{enumerate}[A)]
      \item F ist minimal bei $N_F, N_G, V_F, V_G$ unter der Nebenbedingung
        (*) \\ ODER
      \item %
      \begin{align*}
        \mu (T, v_G) = \mu(T, v_F) \quad (1)
      \end{align*}
      %
      und 
      %
      \begin{align*}
        P(T, v_G) = P(T, v_F) \quad (2)
      \end{align*}
      %
    \end{enumerate} 
    Sie m\"ussen Verfahren B auch in Computeraufgabe 2 verwenden.
    \begin{description}
      \item[L\"osung] Das spezifische Volumen des Gases sollte gr\"o\ss{}er
        als das der Fl\"ussigkeit sein
        %
        \begin{align*}
          v_G \ge v_F > 0
        \end{align*}
        % TODO: Skizze
       Es gibt also nur eine L\"osung f\"ur $T \ge T_c$ n\"ahmlich
       $v_G = v_F$. Damit haben wir das Problem im wesentlichen gel\"ost.
       Damit ist die $v \d{w}$ Gleichung wieder anwendbar.
      
       F\"ur 
       %
       \begin{align*}
         v < v_F^\text{min} \text{ oder } v > v_G ^\text{max}
       \end{align*}
       %
       gibt es nur eine L\"osung mit $v_G = v_F$ und die Van-der-Waals Gleichung
       ist anwendbar.
       %
       \begin{align*}
         v_F^\text{min} < v < v_G^\text{max} ( P_\text{min} < P < P_\text{max})
       \end{align*}
       %
       Es gibt also ein kontinuum von \"osungen zu Gleichung (1). Man kann
       das so schreiben
       %
       \begin{align*}
         v_F(T, v_G) \text{ mit } v_G \in [v_G^\text{min}, v_G^\text{max}]
       \end{align*}
       %
       Aus Gleichung (2) Folgt dann
       %
       \begin{align*}
         \mu(T, v_G) = \mu(T, v_F(T, v_G)) 
       \end{align*}
       %
       ergibt dann $v_G$. Das ist das Verfahren, dass sie in der Computeraufgabe
       verwenden sollten. Es gibt ein altes verfahren, die sogennante Maxwell
       Konstruktion: Man hat ein Praktisches Verfahren entwickelt um den Wert
       von $v_G$ zu bestimmen.
     \item[Maxwell-Konstruktion] 
       % TODO: Skizze
       %
       \begin{align*}
         \int_{v_F(T)}^{v_G(T)} P(T, v) \d{v} = 
         P_M(T) (v_G(T) - v_F(T))
       \end{align*}
       %
       Wir m\"ussen den 1. Hauptsatz f\"ur die freie Enthalpie verwenden
       %
       \begin{align*}
         G(T, P, N) && \d{G} = - S \d{T} + V \d{p} + \mu \d{N}
       \end{align*}
       %
       Nun fragen wir uns, was passiert wenn man bei einer Transformation
       einer isotherme folgt. $ \d{T }$ ist dann Null. Die freie Energie
       ist dann %
       \begin{align*}
         \d{G} = V \d{p} \implies G_G - G_F =
         \int_{P(T, v_F)}^{P(T, v_G)} V(P) \d{P}  =
         \int_{v_F}^{v_G} v \dd{P}{v} \d{v}
       \end{align*}
       %
       Wir k\"onnen nun einfach ein Parielle Integration machen
       %
       \begin{align*}
         G_G - G_F = N[v_G P(T, v_G) - v_F P(T, v_F) - 
         \int_{v_F}^{v_G} P(T, v) \d{v} ] 
       \end{align*}
       %
       Im thermodynamischen Gleichgewicht ist
       %
       \begin{align*}
         G_G = G_F, P(T, v_G) = P(T, v_F) = P_N (T)
       \end{align*}
       %
       Daraus folgt dann auch die Maxwell-Konstruktion.
     \item[Physikalische Zustandsgleichung]
       %
       \begin{align*}
         \tilde{P}(T, v) = \begin{cases}
           P(T, v) & \text{ f\"ur } v < v_F(T) \\
           P_M(T) & \text{ f\"ur } v_F(T) < v < v_G(T) \\
           P(T, v) & \text{ f\"ur } v > v_G(T) \\
         \end{cases} 
       \end{align*}
       %
      Es folgt eine Zusammenfassung der Physik
    \end{description}
  \item[Zusammenfassung]
    %
    \begin{itemize}
      \item F\"ur $T \ge T_c$ gibt es nur eine Phase, und mann kann
        die Van-der-Waals zustandsgleichung verwenden
      \item F\"ur $T < T_c, v < v_F(T)$ gibt es nur eine fl\"ussige Phase, 
        und man kann auch die Van-der-Waals gleichung verwenden.
      \item F\"ur $T < T_c, v > v_G(T)$ Gibt es nur die gasf\"ormige Phase, also
        benutzt man auch die Van-der-Waals Gleichung
      \item F\"ur $T < T_c, v_F(T) < v < v_G(T)$ gibt es 2 Phasen und man kann
        die Van-der-Waals Gleichung nicht verwenden. Es gibt dann konstanten
        Druck, und Chemisches Potential.
        %
        \begin{align*}
          C = \frac{N_F}{N} = \frac{v-G(T) - v}{v_G(T) - v_F(T)}
        \end{align*}
        %
        Ist der Anteil der Materie der in der fl\"ussigen Phase ist.
        % TODO: Skizze
        % TODO: Noch eine Skizze
    \end{itemize}
    %
    % TODO: Skizze
\end{description}


\section*{Ideale Quantengase}
Gase von Bosonen und Fermionen ohne Wechselwirkung.
Sie bilden die Grundlagen f\"ur die Festk\"orperphysik. Wir haben es
in erster N\"aherung mit einem Elektrongas in einem Metall zu tun. In der 
Festk\"orpherphysik gibt es auch die Vibration des Gitters also Phononen.
Beispiele sind 
\begin{itemize}
  \item Photonengas
  \item Ultrakalte Gase mit Bose-Einstein Kondensation
  \item Fermigase
\end{itemize}
\subsection*{Vorlesung 15: Fermi- und Bose-Verteilungen}
\emph{Schwabl Kapitel 4.1}
Wir machen zuerst Quanten-Vielteilchentheorie, das bedeutet wir haben ein 
system von vielen Quantenteilchen $N \gg 1$ welche nicht wechselwirken.
Beginnen wir jedoch mit einem Teilchen um ein paar notationen einzuf\"uhren.
F\"ur diesen Fall haben wir einen Hilbertraum
%
\begin{align*}
  \mathcal{H}^{(1)}_j && \text{ Hamilton-Operator } H_j^{(1)}
\end{align*}
%
Zum beispiel
%
\begin{align*}
  H_j^{(1)} = \frac{\hbar^2}{2m} \Delta_j + V(\vec{r}_j)
\end{align*}
%
Es gibt einteilchen Eigenzust\"ande
%
\begin{align*}
  \Ket{\Phi_\alpha} 
\end{align*}
%
mit Eigenenergien
%
\begin{align*}
  \epsilon_\alpha, \quad \alpha=0,1,2, \ldots
\end{align*}
%
Ordnung
%
\begin{align*}
  \epsilon_0 \le \epsilon_1 \le \epsilon_2 \le \ldots
\end{align*}
%
F\"ur $N$ Teilchen k\"onnen wir den Hilbert-Raum einfach mit einem 
Tensorprodukt bilden.
%
\begin{align*}
  H^{(N)} = \bigotimes_{j=1}^{N} \mathcal{H}_j^{(1)}, &&
  H^{(N)} = \sum_{j=1}^{N} H^{(1)}
\end{align*}
%
Wir k\"onnen sehr leicht die Eienzust\"ande dieses Hamilton-Operators
schreiben. Die $N$-Teilchen  Eigenzust\"ande kann man mit $N$ quantenzahlen
parametrisieren
%
\begin{align*}
  \Ket{\alpha_1 \alpha_2 \alpha_3 \ldots \alpha_N} = \Ket{\alpha_1}
  \otimes \Ket{\alpha_2} \otimes \ldots \otimes \Ket{\alpha_N}
\end{align*}
%
Mit den Eigenenergien
%
\begin{align*}
  E(\alpha_1, \alpha_2, \ldots, \alpha_N) = \sum_{j=1}^{N} \epsilon_{\alpha_j}
\end{align*}
%
Wir k\"onnen das erweitern auf eine beliebeige Zahl von Teilchen, mit
dem sogennanten \emph{Fock-Raum}
%
\begin{align*}
  \mathcal{F} = \bigoplus_{N=0}^{\infty} \mathcal{H}^{(N)}
\end{align*}
%
In jedem Unterraum gibt es den Hamilton-Operator mit der Eigenbasis und den
Eigenenergien. Wir werden nun annehmen, dass die Teilchen alle identisch und
ununterscheidbar sind. Wenn Quantenteilchen identisch sind, dann gibt es jedoch
ein Problem, denn man kann die Teilchen nicht durchnummerieren und ihre Bahnen einzeln bestimmen.
Daraus folgt eine besondere Eigenschaft dieses Zustandes:
Die Funktion sollte nicht von der Nummerierung der Teilchen abh\"angen. Man 
verlangt also, dass wenn man einen Zustand hat der Vertauschungsoperator
folgende Eigenschaft besitzt
%
\begin{align*}
  A_{jl} \Ket{\alpha_1 \alpha_2 \ldots \alpha_N} & = \Ket{\alpha_1 \ldots \alpha_l
  \ldots \alpha_j \ldots \alpha_N} \\
  & = e^{i \varphi_{jl}} \Ket{\alpha_1 \ldots \alpha_j \ldots \alpha_l
\ldots \alpha_N}
\end{align*}
%
Die Anwendung dieses Operators ergibt die Vertauschung von zwei Quantenzahlen.
Wir verlangen, dass die Physik nicht von der Reihenfolge abh\"angt. Es gibt eine
experimentelle Beobachtung f\"ur Bosonen:
%
\begin{align*}
  \varphi_{jl} = 0
\end{align*}
%
F\"ur Fermionen wird beobachtet
%
\begin{align*}
  \varphi_{jl}  = \pi
\end{align*}
%
Das bedeutet es ergibt keinen Sinn in diesem Ganzen Raum zu arbeiten,
sondern wir m\"ussen uns auf Unterr\"aume beschr\"anken die symmetrisch
oder antisymmetrisch sind.
%
\begin{align*}
  \mathcal{H}_S^{(N)} && \mathcal{H}_A^{(N)} \subset \mathcal{H}^{(N)}
\end{align*}
%
%
\begin{align*}
  \Ket{\alpha_1, \ldots \alpha_N}_{S,A} = C_{S,A} \sum_{P}^{} (\pm) P
  \Ket{\alpha_{P(!)} \alpha_{P(3)} \ldots \alpha_{P(N)}}
\end{align*}
%
Je nachdem ob der Zustand symmetrisch oder antisymmetrisch ist gibt es ein
Vorzeichen.  Jeder Zustand hat wieder eine Energie von oben, man hat nur die
Reihenfolge permutiert.  Die Zust\"ande sind also wieder Eigenzust\"ande des
hamilton-Operators mit den Entsprechenden Eigenenergien

%
\begin{align*}
  \mathcal{F} = \bigoplus_{N=0}^{\infty} \mathcal{H}_{S,A}^{(N)}
\end{align*}
%
\subsection*{Besetzungszahldarstellung}
Wir definieren zuerst einen Operator der auf dem Fock-Raum wirkt.
Der Teilchenzahl- oder Besetzungszahloperator
%
\begin{align*}
  \hat{n}_\alpha: \mathcal{F}_{S,A} \to \mathcal{F}_{S,A}
  \hat{n}_\alpha \Ket{\alpha_1, \ldots, \alpha_N}_{S,A} = 
  n_\alpha \Ket{\alpha_1, \ldots ,\alpha_N }_{S,A}
\end{align*}
%
mit $n_\alpha$ der Anzahl der $\alpha_1 = \alpha, \quad j=1, \ldots ,N$
als Besetzungszahl.
Der Teilchenzahloperator
%
\begin{align*}
  \hat{N} = \sum_{\alpha}^{} \hat{n}_{\alpha}
\end{align*}
%
und der Hamilton-Operator
%
\begin{align*}
  H = \sum_{N=0}^{\infty} H^{(N)}  = \sum_{\alpha}^{} \epsilon_\alpha \mathcal{n}_\alpha
\end{align*}
%
%
\begin{align*}
  \Ket{\alpha_1, \ldots, \alpha_N}_{S,A} \leftrightarrow \Ket{n_0 n_1 \ldots}_{S,A} \subset \mathcal{F}_{S,A}
\end{align*}
%
\begin{description}
  \item[Gittermodell] 
    %
    \begin{align*}
      \operatorname{dim} \mathcal{H}^{(1)} & = L && \operatorname{dim} \mathcal{F} = 4^L \\
      \operatorname{dim} \mathcal{H}^{(N_\text{up} N_\text{down})} = 
      \begin{pmatrix}
        L & N_{up} 
      \end{pmatrix} 
      \begin{pmatrix}
        L & N_{down} 
      \end{pmatrix} 
    \end{align*}
    %
\end{description}

%
\begin{align*}
  \hat{N} \Ket{n_0 n_1 \ldots } = N \Ket{ n_0 n_1 \ldots} \text{ mit } N = 
  \sum_{\alpha}^{} n_\alpha \\
  H \Ket{n_0 n_1 \ldots} = E \Ket{n_0 n_1 \ldots } \text{ mit } E = 
  \sum_{\alpha}^{} E_\alpha n_\alpha = E (\alpha_1, \ldots , \alpha_N)
\end{align*}
%
F\"ur Bosonen 
%
\begin{align*}
  0 \le n_\alpha \le \infty \implies \text{ Grundzustand } n_0 = N, \quad
  E_0 = N \epsilon_0
\end{align*}
%
F\"ur Fermionen gilt das Pauli-Prinzip
%
\begin{align*}
  0 \le n_\alpha \le 1 \implies \text{ Grundzustand } n_0 = n_1 - \ldots - n_{N-1} -1,
  \quad n_{\alpha \ge N} = 0 , \quad E_0 = \sum_{\alpha=0}^{N-1} \epsilon_\alpha
\end{align*}
% TODO: skizze
%
\begin{description}
  \item[Zustandsumme]
    \begin{itemize}
      \item Unterscheidbare Teilchen
        %
        \begin{align*}
          Z_N = \trace_{\mathcal{H}^{(N)}} e ^{- \beta H^{(N)}} = \prod_{j=1}^{N}
          \trace_{\mathcal{H}_j^{(1)}} e ^{-\beta H_j ^{(1)}}
        \end{align*}
        %
      \item Identische Quantenteilchen
        %
        \begin{align*}
          Z_{GK} = \trace_{\mathcal{F}_{S,A}} e ^{-\beta \left( \hat{H} - \mu  \hat{N} \right)}
           & = \sum_{}^{} \Braket{\{n_\alpha\} e ^{-\beta \left( \hat{H} - \mu \hat{N} \right)} | \{n_\alpha\} } \\
          \sum_{\{n_\alpha\}}^{} e^{-\beta \left( \sum_{\alpha}^{}
          \epsilon_\alpha n_\alpha - \mu \sum_{\alpha}^{ } n_\alpha \right)}
          = \sum_{n_0}^{} \sum_{n_1}^{} \ldots \prod_{\alpha}^{}
          e^{-\beta (\epsilon_\alpha - \mu )}
         \end{align*}
        %
        %
        \begin{align*}
          Z_{GK} = \prod_{\alpha}^{} \sum_{n_\alpha}^{} e^{-\beta(\epsilon-\mu) n_\alpha}
        \end{align*}
        %
        Bosonen:
        %
        \begin{align*}
          Z_{GK} = \prod_{\alpha}^{} \sum_{n_\alpha = 0}^{ \infty}
          e^{-\beta\left( \epsilon_\alpha - \mu  \right) n_\alpha} 
          = \prod_{\alpha}^{} 
          \frac{1}{1- e ^{-\beta(\epsilon_\alpha - \mu)}}
        \end{align*}
        %
        Diese geomtetrische Reihe ist nur konvergent f\"ur 
        %
        \begin{align*}
          \mu < \epsilon_\alpha \quad\forall\, \alpha
        \end{align*}
        %
        Man kann also das chemische Potential in einem Bose-Gas nicht beliebig 
        w\"ahlen. \\
        Fermionen:
        %
        \begin{align*}
          Z_{GK} &= \prod_{\alpha}^{} \sum_{n_\alpha=0,1}^{} e ^{-\beta(\epsilon_\alpha - \mu)n_\alpha} \\
                 & = \prod_{\alpha}^{} \left( 1 + e^{-\beta(\epsilon_\alpha - \mu)} \right)
        \end{align*}
        %
        
    \end{itemize}
  \item[Mittlere Besetzungszahl]
    %
    \begin{align*}
      \Braket{\hat{n}_\alpha} & = \trace(\hat{n}_\alpha \hat{\rho}) = 
      \frac{1}{Z_{GK}} \trace_{\mathcal{F}_{S,A}} \left( \hat{n}_\alpha
      e ^{-\beta (\hat{H} - \mu \hat{N})}\right) \\
      & = -k_B T \pd{}{\epsilon_\alpha} \ln{Z_{GK}}
    \end{align*}
    %
  \item[Bose-Einstein-Verteilung]
    %
    \begin{align*}
      \Braket{\hat{n}_\alpha} = \frac{1}{e^{-\beta(\epsilon_\alpha - \mu)} - 1}
      \text{ mit } \mu < \epsilon_0 \le \epsilon_\alpha
    \end{align*}
    %
  \item[Fermi-Dirac-Verteilung]
    %
    \begin{align*}
      \Braket{\hat{n}_\alpha} = \frac{1}{e^{\beta(\epsilon_\alpha - \mu)} + 1}
    \end{align*}
    %
\end{description}
Die Anzahl der Teilchen im Einteilchen-Zustand $\Ket{\Phi_\alpha}$
mit Eigenergie $\epsilon_\alpha$
%
\begin{align*}
  \Braket{\hat{n}_\alpha} 
\end{align*}
%


% TODO: Skizze
\subsection*{Zustandsdichte und Photonengas}
\emph{Schwabl Kapitel 4.1, 4.5}
Wir haben Makroskopische Variable von der Form
%
\begin{align*}
  F = \sum_{\alpha}^{} 
  \frac{1}{e^{\beta(\epsilon_\alpha - \mu)} \pm 1}
\end{align*}
%
Die Idee ist
%
\begin{align*}
  F= \int_{-\infty}^{\infty} \d{\epsilon} f(\epsilon) \sum_{\alpha}^{}
  \delta (\epsilon-\epsilon_\alpha) = 
  V \int_{-\infty}^{\infty} \d{\epsilon}f(\epsilon) Z(\epsilon)
\end{align*}
%
Wir definieren die \emph{Einteilchen-Zustandsdichte}
auch \emph{Density of States (DOS)} genannt.
%
\begin{align*}
  Z(\epsilon) = \frac{1}{V} \sum_{\alpha}^{} \delta (\epsilon-\epsilon_\alpha)
\end{align*}
%
\begin{itemize}
  \item $Z(\epsilon)$ ist (stückweise) stetig für $ \lim_{N\to \infty}$
  \item $Z(\epsilon) \d{\epsilon}$ ist die Dichte der Einteilchen-Eigenzustände
    im Intervall $[\epsilon], \epsilon + \d{\epsilon}$.
  \item $V Z(\epsilon) \d{\epsilon}$ ist die Anzahl der Einteilchen-Eigenzustände
    im Intervall $[\epsilon], \epsilon + \d{\epsilon}$.
  \item Eine Folgerung daraus ist
    %
    \begin{align*}
      \int_{-\infty}^{\infty} Z(\epsilon) \d{\epsilon} = \frac{1}{V}
      \operatorname{dim} \mathcal{H}^{(1)}
    \end{align*}
    Um das zu integrieren muss man zuerst $N\to\infty$ schicken. Wir lassen 
    dabei jedoch $\frac{V}{N}$ konstant.
    %
\end{itemize}
Die Zustandsdichte ist keine abstrakte größe, sonder man kann sie auch konkret
messen. Sie ist zum Beispiel relevant für die Wärmeleitfähigkeit eines Metalls.
Wir berechnen jetzt ein Beispiel:
\subsubsection*{Freies Quantengas}
% TODO: Skizze
Es gibt kein Potential, also ist der Hamilton-Operator eines Teilchens
%
\begin{align*}
  H^{(1)} & = - \frac{h^2}{2m} \Delta \\
  H^{(1)} \Psi_\alpha(\vec{r}) & = \epsilon_\alpha \Psi_\alpha (\vec{r})
\end{align*}
%
Die Zustandsdichte hängt nicht von der Randbedingung ab, aber man muss 
trotzdem die Richtige Randebedingung wählen.
\begin{enumerate}[I)]
  \item Offene Randbedingungen
    % TODO: Skizze
    %
    \begin{align*}
      \Psi_k(x) = \sqrt{\frac{2}{L}} \sin{k x}
    \end{align*}
    %
    also
    %
    \begin{align*}
      k & = \frac{\pi}{L}n, \quad n = 1, 2, \ldots \\
      \epsilon(k) & = \frac{\hbar^2}{2 m} k^2
    \end{align*}
    %
  \item Periodische Randbedingungen
    %
    \begin{align*}
      \Psi(x+L) = \Psi(x)
    \end{align*}
    %
    %
    \begin{align*}
      \Psi_k(x) = \frac{1}{\sqrt{L}} e^{i k x} \\
      k = \frac{2 \pi}{L} z, \quad\forall\,z \in \mathbb{Z}
    \end{align*}
    %
    Man kann das mit der Definition der Delta-Funktion ausrechnen, man 
    erhält dann ein sehr leichtes Ergebnis. Wir definieren zuerst eine größe,
    die Anzahl der Eigenzustände mit Energie kleiner als $\epsilon$.
    %
    \begin{align*}
      N(\epsilon) & = \frac{1}{V} & \epsilon(\vec{k}) = \frac{\hbar^2}{2m} \vec{k}^2 < \epsilon \\
                & = \frac{1}{V} \text{ Anzahl der $\vec{k}$ Punkte mit $\abs{\vec{k} < \sqrt{\frac{2 m \epsilon}{ \hbar^2}}}$} \\
                  & = \frac{1}{V} \frac{V_d \left(\sqrt{\frac{2m\epsilon}{ \hbar^2}}\right) } 
      { \left( \frac{2 \pi}{L} \right)^d}
    \end{align*}
    % TODO: Skizze
    %

    \begin{align*}
      V_d(R) = \frac{\pi^{\frac{d}{2} R^d}}{\Gamma\left( \frac{d}{2} + 1 \right)}
    \end{align*}
    %
    %
    \begin{align*}
      Z(\epsilon) = \dd{N}{\epsilon} \begin{cases}
        0 & \text{ für } \epsilon < 0 \\
        (2\pi)^{\frac{d}{2}} \left( \frac{m}{h^2} \right)^{\frac{d}{2}}
        \frac{1}{\Gamma(\frac{d}{2})} \epsilon^{\frac{d}{2} - 1} & \text{ für } \epsilon > 0 \\
      \end{cases} 
    \end{align*}
    %
    %
    Zum beispiel
    \begin{align*}
      d = 1 && Z(\epsilon) = \frac{\sqrt{\frac{2 m }{ \hbar^2}} 1}{\sqrt{\epsilon}} \\
      d = 2 && Z(\epsilon) = \frac{2 \pi m }{ \hbar^2} \\
      d = 3 && Z(\epsilon) = 2 \pi \left( \frac{2m }{ \hbar^2} \right)^{\frac{3}{2}} \sqrt{\epsilon}
    \end{align*}
    %
\end{enumerate} 
\subsubsection*{Photonengas}
Elektromagnetisches Feld in einem Hohlraumresonator
% TODO: Skizze
Die Maxwell-Gleichungen ergeben stehende Wellen
%
\begin{align*}
  w(\vec{k}) = c \abs{\vec{k}}, && \vec{k} = \frac{\pi}{L} \vec{n}, \quad\forall\, n_j \in \N^{\ast}
\end{align*}
%
Quantenmechanisch sind Photonen Bosonen:
%
\begin{align*}
  \epsilon(\vec{k}) = \hbar w (\vec{k}) = \hbar c  \abs{\vec{k}} 
\end{align*}
%
%
\begin{align*}
  H & = \sum_{\vec{k}, \lambda}^{} \epsilon(\vec{k}) \hat{n}_{\vec{k} \lambda} \\
  \lambda & = 1,2 \text{ (Polarisation) }
\end{align*}
%
Wir wollen nun die Energiedichte des Systems berechnen
%
\begin{align*}
  n & = \frac{u}{V}= \frac{\Braket{\hat{H}}}{V} = \frac{1}{V} \sum_{\vec{k}, \lambda}^{} \epsilon(\vec{k})
  \Braket{ \hat{n}_{\vec{k},\lambda} } = \frac{1}{V} \sum_{\vec{k \neq 0 },\lambda}^{} \epsilon(\vec{k}) 
  \frac{1}{e^{\beta \epsilon(\vec{k})} - 1} \\
  & = \sum_{0}^{\infty} \d{\epsilon} * \frac{\epsilon}{e^{\beta \epsilon} - 1} Z(\epsilon) \\
  & = \hbar^2 \int_{0}^{\infty} \d{\omega} * \frac{\omega}{e^{\beta \hbar \omega} - 1} Z(\hbar \omega)
\end{align*}
%
Das ist schon sehr schön, mann muss jetzt noch die Zustandsdichte bestimmen
%
Wir nehmen nun an, dass das System unendlich groß ist und benutzen Periodische Randbedingungen
\begin{align*}
  Z(\hbar\omega) & = \frac{1}{V} \sum_{\vec{k}, \lambda}^{} \delta\left( \hbar \omega - \hbar c \abs{\vec{k}} \right) \\
                 & = \frac{2}{V} * \left( \frac{L}{2\pi} \right)^3 \int_{\R^3}^{} \d{^3 k} \delta(\hbar \omega - \hbar c \abs{\vec{k}}) \\
                 & = \frac{2}{(2\pi)^3}  4 \pi \int_{0}^{\infty} \d{k} k^2 \delta\left( \hbar \omega - c \hbar k \right) \\
                 & = \frac{1}{\pi^2 \hbar c} \left( \frac{\omega}{c} \right)^2
\end{align*}
%
Damit folgt
%
\begin{align*}
  u = \int_{0}^{\infty} \d{\omega} * \frac{\hbar}{\pi^2} \left( \frac{\omega}{c} \right)^3
  \frac{1}{e^{\beta \hbar \omega} -1 }
\end{align*}
%
Der Term im Integranden ist die Spektrale Energiedichte $u(T, \omega)$. \\
% TODO: Skizze
Rayleigh-Jeans Gesetz:
%
\begin{align*}
  \hbar \omega \ll  k_B T && u(T, \omega) = \frac{1}{\pi^2} k_B T * \frac{\omega^2}{c^3} && \text{ Rayleigh-Jeans Gesetz} \\
  \hbar \omega \gg  k_B T && u(T, \omega) ~ e^{-\beta \hbar \omega} && \text{ Wien Gesetz}
\end{align*}
%
%
\begin{align*}
  x = \beta \hbar \omega \implies  && u(T) = (k_B T)^4 * \frac{1}{\pi^2} * \frac{1}{(c \hbar)^3}
  \underbrace{\int_{0}^{\infty} \d{x} * \frac{x^3}{e^x - 1}}_{= b} = a T^4 && \text{ Stefan-Boltzmann-Gesetz }
\end{align*}
%
%
\begin{align*}
  b = \int_{0}^{\infty} \d{x} * \frac{x^4}{\sinh(\frac{x}{2})} = \frac{\pi^4}{15}
\end{align*}
%
%
\begin{align*}
  \implies u = \frac{\pi^2}{15} * \frac{1}{(c \hbar)^3} (k_B T)^4
\end{align*}
%
Beispiel:
%
\begin{align*}
  T_1 \simeq \SI{27}{\kelvin}, V_1 \\
  T_2 = \SI{270}{\kelvin}, V_2 = ? \\
\end{align*}
%
%
\begin{align*}
  U_1 = V_1 u_1 = U_2 = V_2 u_2 \\
  \implies V_2 = V_1 * \frac{u_1}{u_2} = V_1 \left( \frac{T_1}{T_2} \right)^4 = 
  \num{1e-8} V_1
\end{align*}
%



















\subsection*{Bose-Einstein-Kondensation}
Ideales freies Gas von Bosonen mit $s=0,1,2,...$.
Die Dispersion ist
%
\begin{align*}
  \epsilon(\vec{k}) = \frac{\hbar^2}{2m} \vec{k}^2 \ge 0 && s = -S, -S + 1, \ldots , +S
\end{align*}
%
Die Zustandsdichte ist das Produkt mit der Entartung
%
\begin{align*}
  Z(\epsilon \ge 0) = (2 S + 1)\left( \frac{2 \pi m}{\hbar^2}  \right)^{\frac{d}{2}}
  \epsilon^{\frac{d}{2} - 1} * \frac{1}{\Gamma\left( \frac{d}{2} \right)}
\end{align*}
%
Bose-Einstein-Verteilung 
%
\begin{align*}
  \Braket{\hat{n}_{\vec{k},s}} = \Braket{\hat{n}_{\vec{k},s}} = 
  \frac{1}{e^{(\beta(\vec{k}) - \mu)} - 1} && \text{ mit } \mu < 0
\end{align*}
%
Die erste Größe die wir damit zum Beispiel berechnen können ist die mittlere
Teilchenzahl
%
\begin{align*}
  N(T, \mu, V) = \sum_{s, \vec{k}}^{} \Braket{\hat{n}_{\vec{k},s}} = 
  V \int_{0}^{E} \d{\epsilon} Z(\epsilon) * \frac{1}{e^{(\beta(\epsilon - \mu))} - 1}
\end{align*}
%
Daraus folgt
%
\begin{align*}
  N = \frac{V (2 S + 1) }{\lambda_T^d} \int_{0}^{\infty} \d{x} x^{\frac{d}{2} - 1}
  \frac{1}{\frac{1}{z} e^{x} - 1} && \text{ mit } 0 \le Z = e^{\beta\mu} < 1
\end{align*}
%
\begin{description}
  \item[Definition] 
    %
    \begin{align*}
      g_\alpha(z) & = \int_{0}^{+\infty} \d{x} x^{\alpha-1} 
      \frac{1}{\frac{1}{z} e^x - 1} * \frac{1}{\Gamma(\alpha)} && \alpha > 0 \\
                                                               & = \sum_{n=1}^{\infty} 
      \frac{Z^n}{n^\alpha}
    \end{align*}
    Daraus folgt
    %
    \begin{align*}
      N(T, \mu, V) = V * \frac{2S+1}{\lambda_T^d} g_{\frac{d}{2}}(Z)
    \end{align*}
    %
    %
\end{description}
Die innere Energie
%
\begin{align*}
  U(T, \mu, V) = \sum_{\vec{k},s}^{}\epsilon(\vec{k}) \Braket{\hat{n}_{\vec{k},s}}=
  V * \frac{2S+1}{\lambda_T^d} k_B T * \frac{d}{2} g_{\frac{d}{2} + 1}(Z)
\end{align*}
%
Wir brauchen jedoch das großkanonische Potential
%
\begin{align*}
  \Omega(T, \mu, V)  & = + k_B T \sum_{\vec{k},s}^{} \ln{(1-e^{-\beta(\epsilon(\vec{k}) - \mu)})}  \\
                     & = -\frac{2}{d} U(T, \mu, V)
\end{align*}
%
Wir können damit einige allgemeine Ergebnisse herleiten, wie zum Beispiel den Druck
%
\begin{align*}
  P = - \left( \pd{\Omega}{V} \right)_{T, \mu} - \frac{\Omega}{V} = 
  \frac{2}{d} * \frac{U}{V} \implies U = \frac{d}{2} PV
\end{align*}
%
\subsection*{Quantenkorrekturen zum klassischen idealen Gas}
%
\begin{align*}
  Z \ll 1 \iff \abs{\mu} \gg k_B T \implies g_\alpha(z) \simeq Z + Z^{-\alpha} Z^2
  + \mathcal{O}(Z^3) \\
  \implies N = V * \frac{2S+1}{\lambda_T^d}\left(Z + Z^{-\frac{d}{2}} Z^2 + \mathcal{O}(Z^3)\right) \quad (*)
\end{align*}
%
Man kann hier auch eine Entwicklung machen
%
\begin{align*}
  P V = V * \frac{2S+1}{\lambda_T^d} k_B T \left( Z + Z^{-\frac{d}{2} - 1} Z + \mathcal{O}(Z^3) \right)
\end{align*}
%
Man kann damit sehr schön die Änderung der Zustandsgleichung des idealen Gases berechnen
%
\begin{align*}
  \frac{PV }{n k_B T} & = \left( Z + Z^{-\left( \frac{d}{2} + 1 \right)} Z^2 \right)/(Z+ Z^{-\frac{d}{2} Z^2}) + \mathcal{O}(Z^3) \\
                      & = 1 - Z^{-\left( \frac{d}{2} + 1 \right)} Z + \mathcal{O}(Z^3)
\end{align*}
%
Aus $(*)$ folgt
%
\begin{align*}
  Z = \frac{N}{V} \lambda_T^d * \frac{1}{2S +1} = \rho * \frac{\hbar^{d}}{(2 \pi m k_B T)^{\frac{d}{2}}} 
  \frac{1}{2 S + 1}
\end{align*}
%
Wenn $Z \ll 1$
%
\begin{align*}
  \rho \ll 1 && T\to\infty && S \to \infty
\end{align*}
%
%
\begin{align*}
  \mu(T, N, V) = ? \quad \implies g_{\frac{d}{2}}(z) = \frac{\rho \lambda_T^{d} 1}{2 S +1} \quad(+)
\end{align*}
$g_\alpha(z)$ ist monoton steigend und $g_\alpha(0) = 0$. Was passiert nun 
wenn $Z\to1$ ?
%
%
\begin{align*}
  g_{\frac{d}{2}} (z\to1) = \begin{cases}
    +\infty & d = 1,2 \quad \implies \exists z(T, \rho) \quad\forall\, T > 0, \rho > 0 \\
    g_{\frac{3}{2}}(1) \simeq 2.612... & \text{ sonst }\implies \exists z(T, \rho) \quad\forall\, 0 < \rho < \rho_c(T) \text{ oder }
    \quad\forall\, T > T_c(\rho)
  \end{cases} 
\end{align*}
%
mit der kritischen Dichte $\rho_c(T) = \frac{g_z(1}{\lambda_T^3} (2 S + 1)$ 
oder der kritischen Temperatur $T_c(\rho) = \left( \frac{\rho}{(2 S +1)
g_{\frac{3}{2}}(1)}^{\frac{2}{3}} \right) \frac{\hbar^2}{2 \pi m k_B}$.









