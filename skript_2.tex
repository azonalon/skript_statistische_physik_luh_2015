\subsection*{Anwendungen und Rechenmethoden}

\begin{itemize}
    \item Reale Gase und Fl\"ussigkeiten
    \item Ideale Quantengase
    \item Magnetismus
\end{itemize}
\begin{table}[h!]
    \centering
    \begin{tabular}{p{5cm} | p{5cm}}
        Ideales Gas & Reales Gas \\
        \hline
        1 Art von Teilchen & Gemisch (L\"osung) \\
        Punktteilchen & Endliche Ausdehnung, Innere Struktur, Spaltung und Bindung \\
        Keine Wechselwirkung & Wechselwirkung \\
    \end{tabular}
\end{table}
\subsection*{Ideales Molek\"ulgas}
Ideal ist hier im sinne von ``Keine Wechselwirkung'' gemeint. Wir betrachten 
also ein verd\"unntes Gas.
%
\begin{align*}
    \mathcal{H} = \bigotimes_{i=1}^{N} \mathcal{H}_i^{(1)} &&
    H = \sum_{i=1}^{N} H_i^{(1)} && H_i^{(1)} = 
    I_1 \otimes I_2 \otimes \ldots \otimes H_i \otimes I_{n-1} \otimes \ldots 
\end{align*}
%
%
\begin{align*}
    [H_i^{(1)} , H_i^{(1)}]  = 0
\end{align*}
%

%
\begin{align*}
    Z_N & = \trace_{\mathcal{H}^{(N)}} e ^{-\beta H(N)} = 
    \trace_{\mathcal{H}^{(N)}} \prod_{i=1}^{N} e ^{-\beta H_i^{(1)}} \\
    & = \trace_{\mathcal{H}_1^{(1)}}\trace_{\mathcal{H}_2^{(1)}} \ldots \trace_{\mathcal{H}_N^{(1)}} 
    \prod_{i=1}^{N} e^{-\beta H^{1}_i} = \prod_{i=1}^{N} \trace_{\mathcal{H}_i^{(1)}} = 
    \left( \trace_{\mathcal{H}_i^{(1)}} e ^{-\beta H^{(1)}} \right)^{N}
\end{align*}
%
%
\begin{align*}
    F = -k_B T \ln{Z_N} = - k T N \ln{Z_1} && \implies &&  
    \frac{F}{N} = - k_B \ln{Z_1}
\end{align*}
%
\subsection*{Freiheitsgrade eines Molek\"uls}
\begin{itemize}
    \item Bewegung des Schwerpunkts
    \item Rotation
    \item Schwingung (Vibrationen)
    \item Elektronische Anregung
\end{itemize}
Wir nehmen an, dass die Freiheitsgrade s\"amtlich unabh\"angig sind.
%
\begin{align*}
    H^{(1)} & = H_{\text{SP}} + H_{\text{rot}} + H_{\text{vib}} +
    H_{\text{Elek}} \\
    \mathcal{H}^{(1)} & = \mathcal{H}_{\text{sp}} \otimes \mathcal{H}_{\text{rot}} \otimes
    \mathcal{H}_{\text{vib}} \otimes \mathcal{H}_{\text{El}} \\
    Z_1 & = \trace e^{-\beta H^{(1)}} = 
    \left( \trace_{\mathcal{H}_{\text{sp}}} e^{- \beta H_{\text{sp}}} \right)
    \left( \trace_{\mathcal{H}_{\text{rot}}} e^{- \beta H_{\text{rot}}} \right)
    \left( \trace_{\mathcal{H}_{\text{vib}}} e^{- \beta H_{\text{vib}}} \right)
    \left( \trace_{\mathcal{H}_{\text{el}}} e^{- \beta H_{\text{el}}} \right)
\end{align*}
%
Damit ist
%
\begin{align*}
    f = - k_B T \ln{ Z_{\text{sp}}} - k_B T \ln{Z_{\text{rot}}}
- k_B T \ln{Z_{\text{vib}}}
- k_B T \ln{Z_{\text{el}}}
= f_{\text{SP}} + f_{\text{rot}} + f_{\text{vib} } + f_{\text{el}}
\end{align*}
%
Beispiel
% imap <c-s> \text{}<left><left><left><left><left>
%
\begin{align*}
    \frac{C_{V, N}}{N} & = \frac{1}{N} 
    \frac{\delta Q }{ \d{T}} = - \frac{T }{N} \left( \pd{^2 F}{T^2} \right)_{V,N}
    = - T \left( \pd{^2f}{T^2} \right)_{V, N} \\
    & = C_{\text{sp}} + C_{\text{rot}} + C_{\text{vib}} + C_{\text{El}}
\end{align*}
%
%
\begin{align*}
    P = - \left( \pd{F}{V} \right)_{T, N} \simeq - N \left( \pd{f_{\text{sp}}}{V} \right)_{T, N}
\end{align*}
%
% TODO: Diagramm
%
\begin{align*}
    T_0 & = T_F \text{ (Fermi Temperatur) } \\
        & = T_C \text{ (BEC kritische Temperatur)}\\
\end{align*}
%
%
\begin{align*}
    T_0 < \SI{5}{\kelvin} \text{ f\"ur } \ce{He^3}, \ce{He^4}
\end{align*}
%
\subsection*{Rotationsanteil}
F\"ur ein zweiatomiges Moek\"ul aus 2 verschiedenen Atomen zb
\ce{HCl} aber nicht \ce{O2}, \ce{N2}
%
\begin{align*}
    H_\text{rot} = \frac{\vec{L}^2}{2 I}
\end{align*}
%
Mit dem Tr\"agheitsmoment $I = m_\text{reduziert} R^2$
%
\begin{align*}
    \epsilon_{lm} = \frac{\hbar^2}{2 I} l(l+1), \quad\forall\, l \in \N,
    m = -l, -l+1 ,\dotsc, +l
\end{align*}
%
%
\begin{align*}
    \Theta_\text{rot} = \frac{\hbar^2}{I k_B}
\end{align*}
%
%
F\"ur $T \ll \Theta_\text{rot}$:
\begin{align*}
    C_\text{rot} \approx k_B 3 \left( \frac{\Theta_\text{rot}}{T} \right)^2
    e^{- \frac{\Theta_\text{rot}}{T}} \xrightarrow{ T \to 0} 0 
\end{align*}
%
F\"ur hohe Temperaturen $T \gg \Theta_\text{Rot}$ gilt dann
%
\begin{align*}
    C_\text{rot} \approx k_B \left[ 1 + \frac{1}{180} \left( \frac{\Theta_\text{rot}}{T} \right)^2 + \ldots \right] \xrightarrow{T \to \infty} k_B
\end{align*}
%
% TODO: Diagram
Beispiel
%
\begin{align*}
    \ce{H Cl} && \Theta_\text{rot} & = \SI{30}{\kelvin} \\
    \ce{H2} && &= \SI{170}{\kelvin} \\
    \ce{O2} && &= \SI{4}{\kelvin} \\
\end{align*}
%
\subsection*{Vibrationsanteil}
Beispiel: Harmonische Schwingungen eines zweiatomigen Molek\"uls.
Aus der Eigenfrequenz $\omega$ kann man eine Temperaturskala definieren.
%
\begin{align*}
    \Theta_\text{vib} = \frac{\hbar W }{k_B}
\end{align*}
%
Beispiele
%
\begin{align*}
    \ce{H Cl} && \Theta_\text{vib} & = \SI{4e3}{\kelvin} \\
    \ce{H2} && &= \SI{6e3}{\kelvin} \\
    \ce{O2} && &= \SI{2e3}{\kelvin} \\
\end{align*}

%
\begin{align*}
    C_\text{vib} = k_B \left( \frac{\Theta_\text{vib}}{T} \right)^2
    \frac{1}{\left[ 2 \sinh\left( \frac{\Theta_\text{vib}}{2} \right) \right]}
    = \begin{cases}
        \left( \frac{\Theta_\text{vib}}{T} \right)^2 e^{\frac{\Theta_\text{vib}}{T}} \to 0 & T \ll \Theta_\text{vib} \\
        1 - \frac{1}{12}\left( \frac{\Theta_\text{vib}}{T} \right)^2 + \ldots \to 1 & T \gg \Theta_\text{vib}
    \end{cases} 
\end{align*}
%
% TODO: Diagram

\subsection*{Elektronischer Anteil}
Wasserstoffatom
%
\begin{align*}
    \mathcal{E}_n = - \frac{\SI{13.6}{\electronvolt}}{n^2}
\end{align*}
%
Die minimale Anregungsenergie ist
%
\begin{align*}
    \mathcal{E}_2 - \mathcal{E}_1 = \SI{10}{\electronvolt} \approx \SI{1e5}{\kelvin}
\end{align*}
%
% TODO: BIG Diagram
\section*{Reales Klassisches Gas Teil I}
\emph{Schwabl Kapitel 5.3}
\begin{table}[h!]
  \centering
  \begin{tabular}{c c c c c c}
    Klassisch & Verd\"unnt und hohe Temperaturen \\
    Real & Wechselwirkende Teilchen (Atome oder Molek\"ule)
  \end{tabular}
\end{table}
%
\begin{align*}
  Z_K(N) & = \frac{1}{h^{3N}} 
  \frac{1}{N!} \int_{V^N}^{} \d{^{3N}} r \int_{\R^{3N}}^{} e ^{-\beta H (\vec{r}, \vec{p})} \\
  H(\vec{r},\vec{p}) & = \sum_{i=1}^{N} 
  \frac{\vec{P}_1^2}{2m} + \frac{1}{2} \sum_{i,j=1}^{N} w (\abs{\vec{r}_1 - \vec{r}_2})
\end{align*}
%
Dabei ist $w(r)$ ein Zentralpotential.
% TODO: Skizze
%
\begin{align*}
  Z_K(N) = \frac{1}{N! h^{3N}} ( \int_{\R^{3N}}^{} d^{3N}p e^{-\beta \sum_{i}^{} 
  \frac{\vec{P}_i^2}{2m}} 
)\end{align*}
%
Der Konfigurationsanteil ist:
%
\begin{align*}
  Q(T,V, N) = \int_{V^N}^{} \d{^{3N}r} e^{-\beta \sum_{i=1}^{}w (\abs{\vec{r}_i,\vec{p}_i})}
\end{align*}
%
Die thermische Wellenl\"ange ist:
%
\begin{align*}
  \lambda_T = \frac{h}{\sqrt{2 \pi m k_B T}}
\end{align*}
%
F\"ur $w=0$ haben wir ein ideales Gas.
%
\begin{align*}
  Q = V^N
\end{align*}
%
Wenn $w \neq 0$ ist es ein reales Gas, wir k\"onnen dann $Q$ im allgemeinen
nicht berechnen. Es gibt jedoch N\"aherungsverfahren
\begin{itemize}
  \item Virialentwicklung
  \item Molekularfelfdn\"aherung
  \item Computersimulation
  \item Molek\"uldynamik
  \item Monte Carlo
\end{itemize}

Potentiale $w(r)$ f\"ur neutrale Teilchen.
Lennard-Jones-Potential
%
\begin{align*}
  w(r) = 4 \epsilon \left[ \left( \frac{\sigma}{r} \right)^{12} - \left( \frac{\sigma}{r} \right)^6 \right], \quad\forall\, \epsilon, \sigma > 0
\end{align*}
%
% TODO: skizze
Teilchen mit einem harten Kern, also harte Kugeln:
%
\begin{align*}
  w(r) = \begin{cases}
    \infty & r < d \\
    - w_r & d < r \\
    0 r > l \\
  \end{cases} 
\end{align*}
%
\subsection*{Virialentwicklung}
Die Gro\ss{}kanonische Zustandssumme
%
\begin{align*}
  Z_\text{GK} = \sum_{N=0}^{\infty} Z_K(N) e^{\beta \mu N}
\end{align*}
%
%
\begin{align*}
  Z_K(0) = 1, && Z_K(1) = \frac{V}{\lambda_T^3}, && Z_K(2) = ? 
\end{align*}
%
Klassisches Gas: Fugazit\"at $e^{\beta\mu} = y \ll 1$, das bedeutet
$-\mu \gg k_B T$. Mithilfe dieser Annahme k\"onnen wir einfach eine Taylor-Entwicklung
machen
%
\begin{align*}
  Z_\text{GK} = Z_K(0) + Z_K(1) y + Z_K(2)y^2 + \mathcal{O}(y^3) \\
\end{align*}
%
Das bedeutet 
%
\begin{align*}
  \Omega(T, V, \mu) = - k_B T \ln{ Z_\text{GK}} = -k_B T \left[ 
  \frac{V}{\lambda_T^3} y + \left( Z_K(2) - \frac{V}{2 \lambda^3_T} \right) y ^2 
+ \mathcal{O}(y^3)\right]
\end{align*}
%
So erh\"alt man
%
\begin{align*}
  PV = N k_B T \left[ 1+ B \rho + \mathcal{O}(\rho^2) \right] && \rho= \frac{N}{V} \ll 1
\end{align*}
%
$B$ ist der 2. Virialkoeffizient. 
%
\begin{align*}
  B = \frac{V}{2} - \frac{\lambda_T^6 }{V} Z_K(2)
\end{align*}
%
%
\begin{align*}
  Z_K(2) = \frac{1}{\lambda_T^6} 
  \frac{1}{2} \int_{V}^{} \d{^3 r_1} \int_{V}^{} \d{^3r_2} e^{-\beta w(\abs{\vec{r}_1 - \vec{r}_2})}
\end{align*}
%
Daraus folgt dann
%
\begin{align*}
  B = - \frac{1}{2V} \int_{V}^{} \d{^3 r_1} \int_{V}^{} \d{^3 r_2} 
  \left[ e^{-\beta w ( \abs{\vec{r}_1 - \vec{r}_2})} - 1 \right] && \left( \frac{V}{N} \right)^{\frac{1}{3}} \gg \sigma, l
\end{align*}
% TODO: Skizze
%
\begin{align*}
  W \simeq 0 \iff (e^{-\beta w} \approx 1) \quad\forall\, \vec{r}_2 \notin V
  \text{ und } \vec{r}_1 \in V
\end{align*}
%
Wir k\"onnen also das Integral auf den ganzen Raum erweitern und
nicht nur \"uber das Volumen.
%
\begin{align*}
  \vec{r} = \vec{r}_2 - \vec{r}_1 \implies B = - \frac{1}{2V} \int_{V}^{} \d{^3 r_1}
  \int_{\R^3}^{} \d{^3 r} \left[ e^{-\beta w(r)} - 1 \right]
\end{align*}
%
Wir haben einen Kugelsymmetrischen Integranden also verwenden wir Kugelkoordinaten.
%
\begin{align*}
  B = - 2 \pi \int_{0}^{\infty} \d{r} r^2 \left[  e^{-\beta w(r)} - 1 \right]
\end{align*}
%

%
\begin{align*}
  B & = 2 \pi \int_{0}^{d} \d{r} (r^2 e ^{+ \beta w_s} - 1 ) + 0 \\
    & = \frac{2 \pi}{3} d^3 (l^3  - d^3) \left(\e^{\beta w_s} - 1 \right)
  \simeq - \frac{a}{k_B T} + b
\end{align*}
%
mit %
\begin{align*}
  a & = \frac{2\pi}{3} (l^3 - d^3) w_s > 0 \text{ (Anziehende Wechselwirkung)}\\ 
  b & = \frac{4 \cdot 4 \pi}{3} \left( \frac{d}{2} \right)^3 > 0
\end{align*}
%
Man sieht, dass $b$ eigentlich nur 4-mal das Volumen einer Kugel ist.
%
\begin{align*}
  P V = N k_B T \left[ 1 + \left(  - \frac{a}{k_B T} + b \right) 
  \frac{N}{V}\right] + \mathcal{O}\left( \left( \frac{N}{V} \right)^2 \right) && (*)
\end{align*}
%
\subsection*{Van-der-Waals Zustandsgleichung}
%
\begin{align*}
  \left[ P + a\left( \frac{N}{V} \right)^2 \right] ( V - N b) = N k_B T
\end{align*}
%
F\"ur
%
\begin{align*}
  \frac{N}{V} \ll 
  \frac{1}{b} \iff
  \frac{V}{N} \gg b \implies
  (*)
\end{align*}
\subsection*{Harte Kugeln}
%
\begin{align*}
  w(r) = \begin{cases}
    \infty & r < \sigma \\
    0 & r > \sigma \\
  \end{cases} 
\end{align*}
%
%
\begin{align*}
  Q_{HK} = \int_{V'}^{} \d{^{3N}r} = \underset{N \gg 1}{ = } (V - N b)^N
\end{align*}
%
ist das Volumen von $V'$
%
%
\begin{align*}
  V' = \left\{ \{\vec{r}_1\} \in V^N \subset \R^{3N} 
  \text{ mit } \abs{\vec{r}_1 - \vec{r}_{j}} > \sigma \quad\forall\,
 i\neq j = 1, \ldots , N\right\}
\end{align*}
%
$b$ ist gr\"o\ss{}er oder gleich dem Volumen des harten Kerns
$\frac{\pi \sigma^3}{6}$.
Dies ist kein mathematischer Beweis, aber
\begin{itemize}
  \item Exakt in $D=1$
  \item Numerischer Beweis f\"ur $D = 2, 3$
\end{itemize}
Sie m\"ussen das als ein asymptotishes Verhalten
verstehen wenn $N $ sehr gr\"o\ss{} ist. Wenn sie das mit der virialentwicklung
vergleichen dann sehen sie, dass
%
\begin{align*}
  b = 4 \cdot \left( \frac{\pi \sigma^3 }{b} \right)
\end{align*}
%
Das ist Physik, mann kann nicht immer alles mathematisch streng beweisen.  Wir
haben jetzt einen spezialfall diskutiert und werden uns nun allgemeineren
probemen widmen. Wir nehmen an, dass die WEchselwirkung die folgende Struktur hat
%
\begin{align*}
  w(\vec{r}) = \begin{cases}
    w_K(r) > 0 \text{ f\"ur } r  < \sigma \\
    w_K(r) \le 0 \text{ f\"ur } r  > \sigma \\
  \end{cases} 
\end{align*}
%
Wir nehmen an, dass
%
\begin{align*}
  w_K(r) \gg k_B T \gg \abs{ w_F (r)}
\end{align*}
%
\begin{align*}
  w(r) = w_K(r) + w_F(r)
\end{align*}
%
mit 
%
\begin{align*}
  w_K(r > \sigma) &= 0 \\
  w_F(r \le \sigma)& = 0 \\
\end{align*}
%

%
% TODO: Skizze
%
\begin{align*}
  Q = Q_H \frac{\int_{V^N}^{} \d{^{3N}r} e^{-\beta \sum_{i<j}^{} w_K \left( \vec{r}_2  - \vec{r}_1  \right) }   e ^{-\beta \sum_{ i<j }^{} w_F (\vec{r}_2 - \vec{r}_1)}   }{\int_{V^N}^{} \d{^{3N}r} e ^{-\beta \sum_{i=1}^{w_K\left( \vec{r}_1 -\vec{r}_2 \right)}}}
  = Q_\text{HK} \Braket{ e ^{- \beta \sum_{i < j}^{} w_F (\vec{r}_1 - \vec{r}_j)}}_{HK}
\end{align*}
%
$\Braket{ \ldots }_{HK}$ ist der Erwartungswert im Gas harter Kueln
\begin{description}
  \item[Definition] Dichte
    %
    \begin{align*}
      \rho(\vec{r}) = \sum_{i=1}^{N} \delta(\vec{r} - \vec{r}_i) &&
      \int_{\Delta V}^{} \rho (\vec{r}) \d{^3 r} = \sum_{\vec{r}_1 \in
      \Delta V}^{} 1
    \end{align*}
    %
    Dies entspricht der Teilchenzahl in $\Delta V$.
\end{description}

%
\begin{align*}
  \sum_{i < j = 1}^{N} w_F (|\vec{r}_i - \vec{r}_j|) = 
  \frac{1}{2} \int_{V}^{} \d{^3} R_1 \int_{V}^{} \d{^3} R_2 
  w_F (\abs{\vec{R}_1 - \vec{R}_2}) \rho(\vec{R}_1) \rho(\vec{R}_2)
\end{align*}
%
%
\begin{align*}
  Q & = Q_{HK} \Braket{ \exp{
    \left[ - \frac{\beta}{2} \int_{V}^{} \d{^3R_1} \int_{V}^{} \d{^3 R_2}
w_F \left( \abs{\vec{R}_1- \vec{R}_2}  \right) \rho(\vec{R}_1)\rho(\vec{R}_2)\right]}}_{HK} \\
& = Q_{HK}  \exp{
    \left[ - \frac{\beta}{2} \int_{V}^{} \d{^3R_1} \int_{V}^{} \d{^3 R_2}
w_F \left( \abs{\vec{R}_1- \vec{R}_2}  \right) \rho_{MF}(\vec{R}_1)\rho_{MF}(\vec{R}_2)\right]}_{HK}
\end{align*}
%
mit
%
\begin{align*}
  \rho_{MF}(\vec{R}) = \Braket{ \rho(\vec{R})}_{HK}
\end{align*}
%
% TODO: skizze


